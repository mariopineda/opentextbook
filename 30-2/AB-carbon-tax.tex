% !TEX TS-program = pdflatex
% !TEX encoding = UTF-8 Unicode

% This is a simple template for a LaTeX document using the "article" class.
% See "book", "report", "letter" for other types of document.

\documentclass[11pt]{article} % use larger type; default would be 10pt

\usepackage[utf8]{inputenc} % set input encoding (not needed with XeLaTeX)
\usepackage{exercise, hyperref, wasysym}

%%% PAGE DIMENSIONS
\usepackage{geometry} % to change the page dimensions
\geometry{a4paper} % or letterpaper (US) or a5paper or....
% \geometry{margin=2in} % for example, change the margins to 2 inches all round
% \geometry{landscape} % set up the page for landscape
%   read geometry.pdf for detailed page layout information

\usepackage{graphicx} % support the \includegraphics command and options

% \usepackage[parfill]{parskip} % Activate to begin paragraphs with an empty line rather than an indent

%%% PACKAGES
\usepackage{booktabs} % for much better looking tables
\usepackage{array} % for better arrays (eg matrices) in maths
\usepackage{paralist} % very flexible & customisable lists (eg. enumerate/itemize, etc.)
\usepackage{verbatim} % adds environment for commenting out blocks of text & for better verbatim
\usepackage{subfig} % make it possible to include more than one captioned figure/table in a single float
% These packages are all incorporated in the memoir class to one degree or another...

%%% HEADERS & FOOTERS
\usepackage{fancyhdr} % This should be set AFTER setting up the page geometry
\pagestyle{fancy} % options: empty , plain , fancy
\renewcommand{\headrulewidth}{0pt} % customise the layout...
\lhead{}\chead{}\rhead{}
\lfoot{}\cfoot{\thepage}\rfoot{}

%%% SECTION TITLE APPEARANCE
\usepackage{sectsty}
\allsectionsfont{\sffamily\mdseries\upshape} % (See the fntguide.pdf for font help)
% (This matches ConTeXt defaults)

%%% ToC (table of contents) APPEARANCE
\usepackage[nottoc,notlof,notlot]{tocbibind} % Put the bibliography in the ToC
\usepackage[titles,subfigure]{tocloft} % Alter the style of the Table of Contents
\renewcommand{\cftsecfont}{\rmfamily\mdseries\upshape}
\renewcommand{\cftsecpagefont}{\rmfamily\mdseries\upshape} % No bold!

\begin{document}

\begin{table}[ht]
\begin{tabular}{lll}
\hline
Learning Outcome(s): \\
Difficulty: & {\huge \Square \CheckedBox \XBox}  I can not do this yet, I can do some of this, I can do this: \\
\hline
\end{tabular}
\end{table}

Alberta's consumer carbon tax kicked in January 1, 2017. pricing carbon emissions at \$20 per tonne. That rose on January 1, 2018, to \$30 per tonne. In 2007, Alberta began collecting funds from  industrial emitters who released more than 100000 tonnes of greenhouse gases in a year. thoes fees go to the Climate Change and Emissions Management Fund (CCEMF). Here is how much revenue those two taxes have generated in the last two fiscal years, along with the forecast for 2018-19. (Source: Edmonton Journal, March 5, 2019)

\begin{tabular}{ll} 
\hline
\textbf{2016-17} \\
Carbon tax revenue: & \$250 million  \\ 
CCEMF revenue: & \$163 million \\ 
Total: & \$413 million \\ \\

\textbf{2017-18} \\
Carbon tax revenue: & \$1.05 billion  \\ 
CCEMF revenue: & \$251 million \\ 
Total: & \$1.23 billion \\ \\

\textbf{2018-19} \\
Carbon tax revenue: & \$1.3 billion  \\ 
CCEMF revenue: & \$485 million \\ 
Total: & \$1.785 billion \\

\hline
\end{tabular}

\begin{enumerate}
\item[a.] Determine the linear regression function that models the total carbon tax revenue as a function of year, rounding parameters to the nearest hundreth. Sketch the graph, labelling axes with descriptions and units.

\item[b.]  Using the regression function, estimate the total carbon tax revenue, to the nearest dollar, for year 2020.
\end{enumerate}

\end{document}
